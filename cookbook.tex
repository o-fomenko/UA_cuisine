\documentclass[11pt,a4paper]{article}
	\usepackage[T2A]{fontenc}
	\usepackage[english]{babel}
	\usepackage{indentfirst}
	\usepackage{tipa}
	\usepackage{graphicx}
	\usepackage{hyperref}
%	\usepackage{}
%	\usepackage[usestackEOL]{stackengine}
%	\makesavenoteenv{tabular}
	\usepackage[
    	type={CC},
    	modifier={by-nc},
    	version={4.0},
	]{doclicense}
%	\usepackage{cclicenses}
	\graphicspath{{pictures/}}
	\title{\textbf{The Fomenkos Ukrainian cuisine family cookbook}}
	\author{Oleksii Fomenko}
	\date{}

%	\addtolength{\topmargin}{-2cm}
%	\addtolength{\textheight}{2cm}
	
\begin{document}

\section*{Disclaimer}
These recipes are not 'canonical' ones, just like my wife and myself are cooking every day so please add 'in our family' for the every recipe here.
\\

Note: mostly Ukrainian meals are not halal or kosher due to massive pork, lard and meat+milk usage but I'll try to modify recipes or write an alternative ways to cook for non-Ukrainians. I'll really try, I swear :-)
\\
\\
All photos are made by myself using ordinary smartphone, maybe some day I will ask somebody to use good camera for these recipes.
\\
%\cc \bync

\section*{Борщ Borshch \textipa{['bO:rS]}}
Maybe the very first everyone is remaining when hearing anything about 'Ukrainian cuisine'.
Yes, that's the most popular and most controversial meal. It has the only necessary ingredient -- beet root (but not always, 'Green Borshch' has no beet root at all but sorrel or even spinach instead).

\subsection*{Ordinary Borshch}

\doclicenseThis

\end{document}	